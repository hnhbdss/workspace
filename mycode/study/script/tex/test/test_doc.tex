\documentclass{book}
\usepackage{amsmath}
\title{love lili}
\author{dushishuang}
\date{\today}
\begin{document}
\frontmatter
\maketitle
\newpage
\section{this}
A reference to this subsection\footnote{dushishuang love it.}
\label{sec:this} looks like:
‘‘see section \ref{sec:this} on
page \pageref{sec:this}.’’\\
\textit{You can also
\emph{emphasize} text if
it is set in italics,}
\textsf{in a
\emph{sans-serif} font,}
\texttt{or in
\emph{typewriter} style.}\\

\begin{flushleft}
\textsf{
text align left}
\end{flushleft}
\begin{flushright}
text align right
\end{flushright}
\begin{center}
text align center
\end{center}

\begin{enumerate}
\item You can mix the list
environments to your taste:
\begin{itemize}
\item[*] But it might start to
look silly.
\item[-] With a dash.
\end{itemize}
\item Therefore remember:
\begin{description}
\item[Stupid] things will not
become smart because they are
in a list.
\item[Smart] things, though,
can be presented beautifully
in a list.
\end{description}
\end{enumerate}
\verb|dushishuang{enumerate}|
\begin{verbatim} 
dushishuang {asdfads} 
\end{verbatim}
$a^{2}=b^{2} + c^{2}$
comes from \begin{math}\heartsuit\end{math}
\begin{displaymath}
\lim_{n \to \infty}
\sum_{k=1}^n \frac{1}{k^2}
= \frac{\pi^2}{6}
\end{displaymath}
$\lim_{n \to \infty}
\sum_{k=1}^n \frac{1}{k^2}
= \frac{\pi^2}{6}$
\begin{equation}
\forall x \in \mathbf{R}:
\qquad x^{2} \geq 0
\end{equation}
\begin{equation}
x^{2} \geq 0\qquad
\textrm{for all s }x\in\mathbf{R}
\end{equation}
$\lambda,\xi,\pi,\mu,\Phi,\Omega$
$\sqrt{x}$ \qquad
$\sqrt{ x^{2}+\sqrt{y} }$
\qquad $\sqrt[3]{2}$\\[3pt]
$\surd[x^2 + y^2]$
\begin{displaymath}
\vec A\quad\overrightarrow{AB}
\end{displaymath}
\[\lim_{x \rightarrow 0}
\frac{\sin x}{x}=1\]
\begin{displaymath}
\binom{n}{k}\qquad\mathrm{C}_n^k
\end{displaymath}
\begin{displaymath}
\int f_N(x) \stackrel{!}{=} 1
\end{displaymath}
\begin{displaymath}
\sum_{i=1}^{n} \qquad
\int_{0}^{\frac{\pi}{2}} \qquad
\prod_\epsilon
\end{displaymath}
\begin{displaymath}
\sum_{\substack{0<i<n \\ 1<j<m}}
P(i,j) =
\sum_{\begin{subarray}{l}
i\in I\\
1<j<m
\end{subarray}}
Q(i,j)
\end{displaymath}
\begin{displaymath}
1 + \left( \frac{1}{ 1-x^{2} }
\right) ^3
\end{displaymath}
$\Big( (x+1) (x-1) \Big) ^{2}$\\
$\big(\Big(\bigg(\Bigg($\quad
$\big\}\Big\}\bigg\}\Bigg\}$
\quad
$\big\|\Big\|\bigg\|\Bigg\|$
\begin{displaymath}
x_{1},\ldots,x_{n} \qquad
x_{1}+\cdots+x_{n}\qquad
x_{1}+\ddots+x_{n}\qquad
x_{1}+\vdots+x_{n}
\end{displaymath}
\begin{displaymath}
\mathbf{X} =
\left( \begin{array}{ccc}
x_{11} & x_{12} & \ldots \\
x_{21} & x_{22} & \ldots \\
\vdots & \vdots & \ddots
\end{array} \right)
\end{displaymath}
\begin{displaymath}
\left(\begin{array}{c|c}
1 & 2 \\
\hline
3 & 4
\end{array}\right)
\end{displaymath}
\begin{displaymath}
y = \left\{ \begin{array}{ll}
a & \textrm{if $d>c$}\\
b+x & \textrm{in the morning}\\
l & \textrm{all day long}
\end{array} \right.
\end{displaymath}
\begin{displaymath} 
\Gamma_{ij}^{\phantom{ij}k} 
\qquad \textrm{versus} \qquad 
\Gamma_{ij}^{k} 
\end{displaymath} 
\begin{equation}
2^{\textrm{nd}} \quad
2^{\mathrm{nd}}
\end{equation}

\end{document}
